\graphicspath{{img/ch5}}

\section{Upcoming Missions}

Lunar pits have become a focal point for future exploration due to their potential for subsurface access, resource extraction, and habitat construction. Robotic missions will pave the way for human exploration by mapping pit interiors, assessing geological stability, and searching for subsurface cavities.

\subsection{Robotic Missions to Lunar Pits}

Robotic exploration is essential for investigating the geometry, stratigraphy, and resource potential of lunar pits. Several upcoming missions aim to explore the interiors of pits, providing key data to support future human missions.

\subsubsection{NASA's Daedalus Rover}

The **Daedalus rover** is a spherical, autonomous robotic explorer designed to map pit interiors. Using **LIDAR, panoramic cameras, and environmental sensors**, Daedalus can roll along the pit floor, collect data, and generate **3D models of walls and cavities**. It descends into pits using a tethered system, which ensures continuous power and data transmission. The mission aims to confirm the presence of voids and assess subsurface stability \cite{thermal-lunar-pits, newer-thermal}.

\subsubsection{ESA's Tethered Sphere Robot}

The **European Space Agency (ESA)** is developing a similar **tethered exploration system**, which allows the robot to descend into pits while remaining connected to a surface lander. Unlike NASA's free-roaming Daedalus, ESA's version focuses on stable descent and precise mapping of pit interiors. The tether provides **constant power and data relay**, reducing operational risks. This mission will study pit wall composition, stability, and signs of subsurface voids \cite{thermal-lunar-pits}.

\subsubsection{Lunar Reconnaissance Orbiter (LRO) Follow-up Surveys}

While LRO has already provided detailed images and radar scans of pits, future follow-up surveys will focus on **reprocessing LRO datasets with machine learning algorithms**. These methods will improve the detection of thermal anomalies, overhangs, and cavity signatures. **Mini-RF radar** will play a key role in scanning subsurface features, with particular focus on pits like **Mare Tranquillitatis** and **Marius Hills**, where previous radar reflections hinted at potential subsurface cavities \cite{Carrer2024, new-wagner}.

\subsubsection{Private and Commercial Missions}

Private space companies like **Astrobotic** and **ispace** are also planning robotic missions to survey lunar pits, leveraging technologies such as **ground-penetrating radar (GPR)**, LIDAR, and thermal imaging. These missions aim to characterize pit interiors and assess their viability for **resource extraction and human settlement**. By mapping key pits near potential Artemis landing sites, private missions could support future human exploration goals \cite{jsanders-isru}.

---

\subsection{Human Exploration of Lunar Pits}

Future **Artemis missions** aim to explore and assess the habitability of lunar pits. While Artemis III will focus on the lunar south pole, subsequent missions (e.g., Artemis V) are expected to target sites like **Mare Tranquillitatis** and **Marius Hills**, which show strong evidence of subsurface voids and stable overhangs \cite{new-wagner, Carrer2024}.

\subsubsection{Human Descent Technologies}

Exploring the vertical walls of pits presents unique technical challenges. Proposed methods for human descent include:
\begin{itemize}
    \item **Tethered descent systems**: Astronauts could rappel into pits using robotic winches or climbing gear. 
    \item **Legged robotic scouts**: Autonomous robots or climbers could precede astronauts, identifying safe routes for descent.
    \item **Aerial drones**: Drones could provide high-resolution maps of pit interiors and locate points of interest for human exploration \cite{thermal-lunar-pits, newer-thermal}.
\end{itemize}

---

\subsection{Scientific Objectives for Upcoming Missions}

Future missions will address several key scientific objectives. Lunar pits offer direct access to subsurface geology, potential resources, and stable environments for future exploration.

\subsubsection{Access to Stratigraphy}

Pit walls expose the Moon's **stratigraphy**, revealing stacked layers of ancient lava flows. By analyzing these layers, scientists can reconstruct the **volcanic history of the Moon** and investigate how lava tube systems evolved. Stratigraphic differences at pits like **Mare Tranquillitatis** and **Marius Hills** suggest regional variations in volcanic activity \cite{new-wagner}.

\subsubsection{Detection of Subsurface Cavities}

A major goal of upcoming missions is the detection of **subsurface cavities** beneath pit floors. Cavity detection will rely on **ground-penetrating radar (GPR)**, **gravitational anomaly analysis**, and **Mini-RF radar scans**. Observations of radar reflections at **Mare Tranquillitatis** have already revealed possible subsurface voids \cite{Carrer2024}. By confirming the presence of accessible voids, missions could identify potential sites for resource extraction and habitation.

\subsubsection{Search for Volatiles and Water Ice}

Pits located near **permanently shadowed regions (PSRs)** may act as **traps for water ice and volatiles**. The interior geometry of pits, particularly those with overhanging structures, creates localized cold environments where volatiles could accumulate. Missions will use **infrared spectrometers** and radar to detect water ice, which is essential for **In-Situ Resource Utilization (ISRU)**. Extracted water could be converted into hydrogen and oxygen, serving as **rocket fuel and life support** for human missions \cite{jsanders-isru}.

---

\subsection{Summary of Key Missions and Technologies}

\begin{itemize}
    \item **NASA's Daedalus Rover** — Spherical robotic explorer for 3D mapping and void detection \cite{thermal-lunar-pits}.
    \item **ESA's Tethered Sphere Robot** — Tethered robotic explorer focused on stability and power efficiency \cite{thermal-lunar-pits}.
    \item **Artemis V Missions** — Human-led missions to survey and explore pits using tethered descent and climbers \cite{thermal-lunar-pits}.
    \item **LRO Follow-up Surveys** — Advanced radar and machine learning analysis to detect subsurface cavities \cite{Carrer2024, new-wagner}.
    \item **Private Missions** — Robotic scouts from companies like Astrobotic and ispace to explore potential Artemis landing sites \cite{jsanders-isru}.
\end{itemize}