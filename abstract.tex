\begin{abstract}
Lunar pits, discovered through high-resolution imagery from missions such as the Lunar Reconnaissance Orbiter (LRO) and SELENE, are transformative geological features that provide unparalleled access to the Moon's subsurface. These pits reveal ancient stratigraphic layers, offering insights into volcanic processes and the evolution of the lunar crust. Beyond their geological significance, they present practical opportunities as stable, sheltered environments that could support future exploration, habitation, and resource utilization. This work consolidates current understanding of lunar pit morphology, formation mechanisms, thermal properties, and spatial distribution, integrating key findings from radar imaging, thermal modeling, and gravitational studies. With their potential to serve as gateways to subsurface voids and natural laboratories for planetary science, lunar pits are emerging as critical focal points in the pursuit of sustainable lunar exploration and habitation strategies.
\end{abstract}